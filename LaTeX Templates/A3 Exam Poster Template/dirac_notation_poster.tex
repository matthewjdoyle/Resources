% ========================================
% DIRAC NOTATION QUANTUM MECHANICS POSTER
% ========================================
% 
% MODE SELECTION:
% For LIGHT MODE: \documentclass{a3cheatsheet}
% For DARK MODE:  \documentclass[darkmode]{a3cheatsheet}
%
% Simply uncomment the desired mode below:

% Light Mode (default)
\documentclass{a3cheatsheet}

% Dark Mode (uncomment the line below and comment out the light mode line above)
% \documentclass[darkmode]{a3cheatsheet}

% Document information
\title{Dirac Notation in Quantum Mechanics}
\author{Quantum Physics Reference}
\date{\today}

\begin{document}

% Title with optional logo
\cheatsheettitle{Dirac Notation}{Complete Reference for Quantum Mechanics}

\begin{multicols}{3}

% ======= SECTION 1: BASICS =======
\section{Fundamental Concepts}

\begin{infobox}[Dirac Notation Definition]
\textbf{Bra-Ket Notation} (Dirac notation) is a standard notation for describing quantum states and operations in quantum mechanics.
\end{infobox}

\keypoint{Three Key Components:}
\begin{itemize}
    \item \textbf{Ket}: $|\psi\rangle$ - column vector representing a quantum state
    \item \textbf{Bra}: $\langle\phi|$ - row vector (complex conjugate transpose)
    \item \textbf{Bracket}: $\langle\phi|\psi\rangle$ - inner product (probability amplitude)
\end{itemize}

\begin{formulabox}[Basic Definitions]
\begin{align}
\text{Ket:} \quad |\psi\rangle &\equiv \begin{pmatrix} c_1 \\ c_2 \\ \vdots \\ c_n \end{pmatrix} \\
\text{Bra:} \quad \langle\psi| &\equiv (c_1^*, c_2^*, \ldots, c_n^*) \\
\text{Inner Product:} \quad \langle\phi|\psi\rangle &= \sum_i \phi_i^* \psi_i
\end{align}
\end{formulabox}

\begin{notebox}[Normalization]
For normalized states:
$$\langle\psi|\psi\rangle = 1$$
This ensures the total probability equals unity.
\end{notebox}

% ======= SECTION 2: OPERATIONS =======
\section{Mathematical Operations}

\subsection{Linear Combinations}

\begin{formulabox}[Superposition Principle]
Any quantum state can be written as:
$$|\psi\rangle = \sum_i c_i |i\rangle$$
where $|i\rangle$ are basis states and $c_i$ are complex coefficients.
\end{formulabox}

\subsection{Operators}

\begin{infobox}[Operator Action]
Operators $\hat{A}$ act on kets:
$$\hat{A}|\psi\rangle = |\phi\rangle$$
\end{infobox}

\begin{formulabox}[Expectation Values]
The expectation value of operator $\hat{A}$:
$$\langle\hat{A}\rangle = \langle\psi|\hat{A}|\psi\rangle$$
\end{formulabox}

\tip{Operators are represented by hats (e.g., $\hat{H}$, $\hat{p}$, $\hat{x}$)}

\subsection{Outer Products}

\begin{formulabox}[Projection Operators]
The outer product creates projection operators:
$$|\psi\rangle\langle\phi| = \hat{P}_{\psi,\phi}$$
Special case - projector onto state $|\psi\rangle$:
$$\hat{P}_\psi = |\psi\rangle\langle\psi|$$
\end{formulabox}

% ======= SECTION 3: COMMON STATES =======
\section{Standard Quantum States}

\subsection{Spin-1/2 States}

\begin{formulabox}[Pauli Basis States]
\begin{align}
|+\rangle &= \begin{pmatrix} 1 \\ 0 \end{pmatrix}, \quad |-\rangle = \begin{pmatrix} 0 \\ 1 \end{pmatrix} \\
|+_x\rangle &= \frac{1}{\sqrt{2}}\begin{pmatrix} 1 \\ 1 \end{pmatrix}, \quad |-_x\rangle = \frac{1}{\sqrt{2}}\begin{pmatrix} 1 \\ -1 \end{pmatrix} \\
|+_y\rangle &= \frac{1}{\sqrt{2}}\begin{pmatrix} 1 \\ i \end{pmatrix}, \quad |-_y\rangle = \frac{1}{\sqrt{2}}\begin{pmatrix} 1 \\ -i \end{pmatrix}
\end{align}
\end{formulabox}

\subsection{Position and Momentum}

\begin{formulabox}[Continuous Basis States]
\begin{align}
\text{Position:} \quad |x\rangle &\text{ - eigenstate of } \hat{x} \\
\text{Momentum:} \quad |p\rangle &\text{ - eigenstate of } \hat{p} \\
\langle x|p\rangle &= \frac{1}{\sqrt{2\pi\hbar}} e^{ipx/\hbar}
\end{align}
\end{formulabox}

\begin{notebox}[Wave Functions]
The wave function is the position representation:
$$\psi(x) = \langle x|\psi\rangle$$
\end{notebox}

% ======= SECTION 4: IMPORTANT RELATIONS =======
\section{Key Relations \& Identities}

\subsection{Completeness Relations}

\begin{formulabox}[Resolution of Identity]
For discrete basis:
$$\sum_i |i\rangle\langle i| = \hat{I}$$
For continuous basis:
$$\int |x\rangle\langle x| dx = \hat{I}$$
\end{formulabox}

\subsection{Orthogonality}

\begin{formulabox}[Orthonormal Basis]
\begin{align}
\langle i|j\rangle &= \delta_{ij} \quad \text{(discrete)} \\
\langle x|x'\rangle &= \delta(x-x') \quad \text{(continuous)}
\end{align}
\end{formulabox}

\warning{Be careful with normalization factors in continuous bases!}

% ======= SECTION 5: DYNAMICS =======
\section{Time Evolution}

\subsection{Schrödinger Equation}

\begin{formulabox}[Time-Dependent Schrödinger Equation]
$$i\hbar\frac{\partial}{\partial t}|\psi(t)\rangle = \hat{H}|\psi(t)\rangle$$
where $\hat{H}$ is the Hamiltonian operator.
\end{formulabox}

\subsection{Time Evolution Operator}

\begin{formulabox}[Unitary Evolution]
$$|\psi(t)\rangle = \hat{U}(t,t_0)|\psi(t_0)\rangle$$
where $\hat{U}(t,t_0) = e^{-i\hat{H}(t-t_0)/\hbar}$
\end{formulabox}

\begin{infobox}[Properties of $\hat{U}$]
\begin{itemize}
    \item Unitary: $\hat{U}^\dagger\hat{U} = \hat{I}$
    \item $\hat{U}(t_0,t_0) = \hat{I}$
    \item $\hat{U}(t_2,t_0) = \hat{U}(t_2,t_1)\hat{U}(t_1,t_0)$
\end{itemize}
\end{infobox}

% ======= SECTION 6: MEASUREMENT =======
\section{Quantum Measurement}

\subsection{Born Rule}

\begin{formulabox}[Probability of Measurement]
Probability of measuring eigenvalue $a_n$ of operator $\hat{A}$:
$$P(a_n) = |\langle a_n|\psi\rangle|^2$$
where $|a_n\rangle$ is the corresponding eigenstate.
\end{formulabox}

\subsection{Post-Measurement State}

\begin{formulabox}[State Collapse]
After measuring $a_n$, the state becomes:
$$|\psi'\rangle = \frac{\hat{P}_n|\psi\rangle}{\sqrt{\langle\psi|\hat{P}_n|\psi\rangle}}$$
where $\hat{P}_n = |a_n\rangle\langle a_n|$
\end{formulabox}

\keypoint{Measurement Types:}
\begin{itemize}
    \item \textbf{Projective}: Uses projection operators
    \item \textbf{POVM}: Positive Operator-Valued Measure
    \item \textbf{Weak}: Minimal disturbance measurements
\end{itemize}

% ======= SECTION 7: ADVANCED TOPICS =======
\section{Advanced Concepts}

\subsection{Tensor Products}

\begin{formulabox}[Composite Systems]
For two systems A and B:
$$|\psi_{AB}\rangle = |\psi_A\rangle \otimes |\psi_B\rangle$$
In general:
$$|\psi_{AB}\rangle = \sum_{i,j} c_{ij}|i_A\rangle \otimes |j_B\rangle$$
\end{formulabox}

\subsection{Entanglement}

\begin{formulabox}[Bell States]
\begin{align}
|\Phi^+\rangle &= \frac{1}{\sqrt{2}}(|00\rangle + |11\rangle) \\
|\Phi^-\rangle &= \frac{1}{\sqrt{2}}(|00\rangle - |11\rangle) \\
|\Psi^+\rangle &= \frac{1}{\sqrt{2}}(|01\rangle + |10\rangle) \\
|\Psi^-\rangle &= \frac{1}{\sqrt{2}}(|01\rangle - |10\rangle)
\end{align}
\end{formulabox}

\begin{notebox}[Entanglement Criterion]
A state is entangled if it cannot be written as:
$$|\psi\rangle \neq |\psi_A\rangle \otimes |\psi_B\rangle$$
\end{notebox}

% ======= SECTION 8: USEFUL FORMULAS =======
\section{Essential Formulas}

\subsection{Commutation Relations}

\begin{formulabox}[Canonical Commutators]
\begin{align}
[\hat{x}, \hat{p}] &= i\hbar \\
[\hat{L}_i, \hat{L}_j] &= i\hbar\epsilon_{ijk}\hat{L}_k \\
[\hat{a}, \hat{a}^\dagger] &= 1 \quad \text{(harmonic oscillator)}
\end{align}
\end{formulabox}

\subsection{Uncertainty Relations}

\begin{formulabox}[Heisenberg Uncertainty Principle]
For any two operators $\hat{A}$ and $\hat{B}$:
$$\Delta A \cdot \Delta B \geq \frac{1}{2}|\langle[\hat{A},\hat{B}]\rangle|$$
Most famous case:
$$\Delta x \cdot \Delta p \geq \frac{\hbar}{2}$$
\end{formulabox}

\tip{The uncertainty principle is a fundamental feature of quantum mechanics, not a limitation of measurement!}

\section{References \& Resources}

\begin{infobox}[Recommended Reading]
\textbf{Classic Textbooks:}
\begin{itemize}
    \item Griffiths - Introduction to Quantum Mechanics
    \item Sakurai - Modern Quantum Mechanics
    \item Nielsen \& Chuang - Quantum Computation
\end{itemize}

\textbf{Key Papers:}
\begin{itemize}
    \item Dirac - The Principles of Quantum Mechanics
    \item Born - Statistical Interpretation
    \item Bell - On the Einstein-Podolsky-Rosen Paradox
\end{itemize}
\end{infobox}

\begin{notebox}[Quick Reference]
\textbf{Common Notation:}
\begin{itemize}
    \item $\hbar = \frac{h}{2\pi}$ - reduced Planck constant
    \item $\hat{H}$ - Hamiltonian (energy operator)
    \item $\hat{U}$ - unitary time evolution operator
    \item $\delta_{ij}$ - Kronecker delta
    \item $\delta(x)$ - Dirac delta function
\end{itemize}
\end{notebox}

\end{multicols}

\end{document} 