% =============================================================================
% Python Technical Reference Sheet
% Landscape A4 format - Perfect for Technical Interviews
% =============================================================================

\documentclass[11pt]{article}
\usepackage{style}

% Document metadata
\title{Python Reference Sheet}
\author{Technical Interview Guide}
\date{\today}

\begin{document}

% Header with technology name and version
% \techheader[Python 3.11]{Python}

% Quick Reference Section
\quickref{
  \begin{tabular}{llll}
    \textbf{Type} & \textbf{Example} & \textbf{Type} & \textbf{Example} \\
    \hline
    \datatype{int} & \code{42} & \datatype{float} & \code{3.14} \\
    \datatype{str} & \code{"hello"} & \datatype{bool} & \code{True} \\
    \datatype{list} & \code{[1, 2, 3]} & \datatype{tuple} & \code{(1, 2, 3)} \\
    \datatype{dict} & \code{\{"a": 1\}} & \datatype{set} & \code{\{1, 2, 3\}}
  \end{tabular}
}

% Basic Syntax Section
\section{Basic Syntax}

\begin{multicols}{3}
  \syntaxbox{Variable Assignment}{
    \code{name = value}\\
    \code{x, y = 1, 2}\\
    \code{a = b = c = 0}
  }
  
  \syntaxbox{Function Definition}{
    \code{def func(args):}\\
    \code{    return value}\\
    \code{def func(a, b=0, *args):}
  }
\end{multicols}

% Control Structures
\section{Control Structures}

\begin{multicols}{3}
  \subsection{If Statement}
  \begin{moderncode}
if condition:
    statement
elif other_condition:
    statement
else:
    statement
  \end{moderncode}
  
  \subsection{For Loop}
  \begin{moderncode}
for item in iterable:
    statement
    
for i in range(10):
    print(i)
  \end{moderncode}
\end{multicols}

\begin{multicols}{3}
  \subsection{While Loop}
  \begin{moderncode}
while condition:
    statement
    if break_condition:
        break
  \end{moderncode}
  
  \subsection{List Comprehension}
  \begin{moderncode}
\# Basic
[x for x in range(10)]

\# With condition
[x for x in range(10) if x % 2 == 0]

\# Nested
[(x, y) for x in [1,2] for y in [3,4]]
  \end{moderncode}
\end{multicols}

% Data Structures
\section{Data Structures}

\begin{multicols}{3}
  \subsection{Lists}
  \begin{itemize}
    \item \code{list.append(x)} - Add item to end
    \item \code{list.insert(i, x)} - Insert at position
    \item \code{list.remove(x)} - Remove first occurrence
    \item \code{list.pop([i])} - Remove and return item
    \item \code{list.sort()} - Sort in place
    \item \code{list.reverse()} - Reverse in place
  \end{itemize}
  
  \subsection{Dictionaries}
  \begin{itemize}
    \item \code{dict.get(key, default)} - Get with default
    \item \code{dict.update(other)} - Update with other dict
    \item \code{dict.pop(key)} - Remove and return value
    \item \code{dict.keys()} - Get all keys
    \item \code{dict.values()} - Get all values
    \item \code{dict.items()} - Get key-value pairs
  \end{itemize}
\end{multicols}

% Common Methods
\section{Common Methods}

\begin{multicols}{3}
  \subsection{String Methods}
  \method{str.split(sep)}{Split string by separator}
  \method{str.join(iterable)}{Join iterable with string}
  \method{str.strip()}{Remove whitespace}
  \method{str.replace(old, new)}{Replace substring}
  \method{str.startswith(prefix)}{Check prefix}
  \method{str.endswith(suffix)}{Check suffix}
  \method{str.upper()}{Convert to uppercase}
  \method{str.lower()}{Convert to lowercase}
  
  \subsection{List Methods}
  \method{list.append(x)}{Add item to end}
  \method{list.extend(iterable)}{Extend with iterable}
  \method{list.insert(i, x)}{Insert at index}
  \method{list.remove(x)}{Remove first occurrence}
  \method{list.pop([i])}{Remove and return item}
  \method{list.index(x)}{Find index of item}
  \method{list.count(x)}{Count occurrences}
  \method{list.sort()}{Sort in place}
\end{multicols}

% Key Concepts
\section{Key Concepts}

\begin{multicols}{3}
  \conceptbox{List vs Tuple}{
    \textbf{Lists:} Mutable, use \code{[]}\\
    \textbf{Tuples:} Immutable, use \code{()}\\
    Tuples are faster and use less memory
  }
  
  \conceptbox{Shallow vs Deep Copy}{
    \textbf{Shallow:} \code{list.copy()}\\
    \textbf{Deep:} \code{copy.deepcopy()}\\
    Deep copy creates new nested objects
  }
\end{multicols}

\begin{multicols}{3}
  \conceptbox{Generator Expressions}{
    \code{(x for x in range(10))}\\
    Memory efficient, lazy evaluation\\
    Use parentheses, not brackets
  }
  
  \conceptbox{Context Managers}{
    \code{with open('file.txt') as f:}\\
    Automatic resource management\\
    Ensures cleanup even with exceptions
  }
\end{multicols}

% Common Patterns
\section{Common Patterns}

\begin{multicols}{3}
  \subsection{Error Handling}
  \begin{moderncode}
try:
    risky_operation()
except ValueError as e:
    handle_error(e)
except Exception as e:
    handle_generic(e)
finally:
    cleanup()
  \end{moderncode}
  
  \subsection{File Operations}
  \begin{moderncode}
\# Read file
with open('file.txt', 'r') as f:
    content = f.read()

\# Write file
with open('file.txt', 'w') as f:
    f.write('content')
  \end{moderncode}
\end{multicols}

\begin{multicols}{3}
  \subsection{Lambda Functions}
  \begin{moderncode}
\# Simple lambda
lambda x: x * 2

\# With filter
filter(lambda x: x > 0, numbers)

\# With map
map(lambda x: x**2, numbers)
  \end{moderncode}
  
  \subsection{Decorators}
  \begin{moderncode}
def decorator(func):
    def wrapper(*args, **kwargs):
        \# Do something before
        result = func(*args, **kwargs)
        \# Do something after
        return result
    return wrapper

@decorator
def function():
    pass
  \end{moderncode}
\end{multicols}

% Important Notes
% \importantnote{
%   \textbf{Indentation:} Use 4 spaces, not tabs\\
%   \textbf{Names:} Use snake\_case for variables and functions\\
%   \textbf{Classes:} Use CamelCase for class names\\
%   \textbf{Constants:} Use UPPER\_CASE for constants
% }

% Footer with additional information
\vfill
\begin{center}
  \textcolor{darkgray}{\small Python 3.11 Reference \quad | \quad Technical Interview Guide}
\end{center}

\end{document}
